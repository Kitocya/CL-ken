\documentclass[a4paper,11pt]{ltjsarticle}
\usepackage{nomal_preamble}
\usepackage{url,anyfontsize}

\title{生命化学学術研究会の現状}
\author{\today}
\date{}

\makeatletter
\makeatother

\begin{document}

\pagestyle{fancy}
\lhead{\;生命化学学術研究会}
\rhead{}
\cfoot{\thepage}

\maketitle

\tableofcontents

\hspace*{12pt}

目次をタップすると,対応したページに遷移します.

\clearpage

\section{顧問}
\begin{itemize}
    \item 荻原 淳(発酵化学研究室/生化,バイオ)
\end{itemize}

\section{幹部}

\begin{enumerate}
    \item サークル長
    \begin{itemize}
        \item 斎藤 伶穏(生化/3年)
        \item 後継:庄司 湧人(バイオ/2年)の可能性
    \end{itemize}
    \item 副サークル長
    \begin{itemize}
        \item 松井 花歩(生化/3年)
        \item (以下,3年のサークル長と副サークル長をまとめて長副と呼ぶ.)
        \item 後継:未定(恐らく,中島 悠理(バイオ/2年)の可能性)
    \end{itemize}
    \item 実験長
    \begin{itemize}
        \item 畑山 日陽里(生化/3年)
        \item 後継:寺谷 優輝(バイオ/2年)
    \end{itemize}
    \item 会計長
    \begin{itemize}
        \item 鶴岡 一花(生化/3年)
        \item 後継:未定(恐らく,今泉 優哉(環境/2年)の可能性)
    \end{itemize}
\end{enumerate}

\section{活動内容}

\begin{itemize}
    \item 実験(現状,年に2回)
    \begin{itemize}
        \item 新入生歓迎会の時期(以下,春期実験と呼ぶ,)に1回行い,新入生でもできる簡単な実験を行う.
        \item 夏季休暇中に2回目を行い,春期実験で出てきた課題を元に実験(研究)を行う.
        \item 学部祭で教室を1つ借りて,ポスター発表をする.
        \item 11-12月に行われる学術部連盟でスライドを用いて発表を行う.(要旨集の作成も行う.)
    \end{itemize}
    \item 本質的でない活動(現状のサークルはこちらがメインの活動になっています.( )内は本音の部分です.)
    \begin{itemize}
        \item 学部祭の出店(昨年,今年はかなりひどいものでした.\footnote{以下のセクションで詳しく明記します.})
        \item 課外活動(ほぼ観光しかしてないです.)
        \item 長期休みの合宿(サークル長たちの旅行に同行しただけ.(長副以外の3年は参加していないです.))
    \end{itemize}
\end{itemize}

\end{document}