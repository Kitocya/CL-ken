\documentclass[a4paper,11pt]{ltjsarticle}
\usepackage{nomal_preamble}
\usepackage{url,anyfontsize}

\title{生命化学学術研究会の現状}
\author{\today}
\date{}

\makeatletter
\makeatother

\begin{document}

\pagestyle{fancy}
\lhead{\;生命化学学術研究会}
\rhead{}
\cfoot{\thepage}

\maketitle

\tableofcontents

\hspace*{12pt}

目次をタップすると,対応したページに遷移します.

\clearpage

\section{概要}

\subsection{顧問および幹部}

\begin{enumerate}
    \item 顧問
    \begin{itemize}
        \item 荻原 淳(発酵化学研究室/生化,バイオ)
    \end{itemize}
    \item サークル長
    \begin{itemize}
        \item 斎藤 伶穏(生化/3年)
        \item 後継:庄司 湧人(バイオ/2年)の可能性
    \end{itemize}
    \item 副サークル長
    \begin{itemize}
        \item 松井 花歩(生化/3年)
        \item (以下,3年のサークル長と副サークル長をまとめて長副と呼ぶ.)
        \item 後継:未定(恐らく,中島 悠理(バイオ/2年)の可能性)
    \end{itemize}
    \item 実験長
    \begin{itemize}
        \item 畑山 日陽里(生化/3年)
        \item 後継:寺谷 優輝(バイオ/2年)
    \end{itemize}
    \item 会計長
    \begin{itemize}
        \item 鶴岡 一花(生化/3年)
        \item 後継:未定(恐らく,今泉 優哉(環境/2年)の可能性)
    \end{itemize}
    \item 全体の人数
    \begin{itemize}
        \item[3年] 生化:8人
        \item[2年] バイオ:3人,森林:2人,環境:1人,動物:1人
        \item[1年] バイオ:3人
    \end{itemize}
\end{enumerate}

\subsection{活動内容}

\begin{itemize}
    \item 実験(現状,年に2回)
    \begin{itemize}
        \item 新入生歓迎会の時期(以下,春期実験と呼ぶ,)に1回行い,新入生でもできる簡単な実験を行う.
        \item 夏季休暇中に2回目を行い,春期実験で出てきた課題を元に実験(研究)を行う.
        \item 学部祭で教室を1つ借りて,ポスター発表をする.
        \item 11-12月に行われる学術部連盟でスライドを用いて発表を行う.(要旨集の作成も行う.)
    \end{itemize}
\clearpage
    \item 本質的でない活動(現状のサークルはこちらがメインの活動になっています.( )内は本音の部分です.)
    \begin{itemize}
        \item 学部祭の出店(昨年,今年はかなりひどいものでした.\footnote{以下のセクションで詳しく明記します.})
        \item 課外活動(ほぼ観光しかしてないです.)
        \item 長期休みの合宿(長副の旅行に同行しただけです.(3年の参加者は長副のみ.))
    \end{itemize}
\end{itemize}

\section{論理学に反する事象}

\subsection{学部祭1日目「休業」事件}

アイコンにしているので事の経緯は何となくわかっていると思うが,一応記載しておきます.
まず,当初は1年が5-6人ほど入部をしました.しかしながら,学部祭の役職決めの際に人の意見を聞かずに強制全員参加という通達があり,3人ほど辞めました.
それにも関わらず,その辞めた人を一切把握しようとせず4月に作成したままの名簿のまま役職を割り振りました.

重要な役割につけた人に関しては,長副から連絡が行き講習会に参加するように促され(Wさんもほぼ同様です.半分恐喝だったらしいです.)辞めたのにも関わらず参加させられるという意味不明なことが起こりました.
当然,参加するわけもなく無断欠席となりいくつかペナルティーをもらったそうです.
ここでも問題が起こりますが,誰が無断欠席をしたのかも把握していません.(音信不通になったのにも関わらず.)

学部祭,前日に長副以外の先輩たち(以下,女性陣と呼ぶ.)が大変なことに気づき(今まで,他の活動で手一杯でした.),長副に連絡(この時にWさんが辞める.)しました.
この際,「何故,辞めた人(音信不通な人)に重要な役割を担わせるのか」「自分自身は,何故役職についていないのか」と矛盾が生じ過ぎててこの時点でもう諦めました.

\subsection{「防火防災訓練」参加したのにしなかったことにされた事件}

これに関しては,参加しなければならないという通知は,1回も来ていなく1週間前にあるかも?,当日の10分前に「集合は〇〇です.」と通達が来るという遅さです.
こんなので参加するわけもなく,参加したのは3人ほどです.実は,実際に参加したのは6人ほどで受付をしないといけないということで参加したのに不当な扱いを受けた人が数人いるいます.
言葉足らずかつ,日本語が通じない,解読できない,上から目線などの要因から意思疎通ができなかったのが真実です.

\subsection{その他}

今年,昨年も多くの問題が生じてきましたが,女性陣3人によって守り切ったという感じです.他の事象が知りたい場合教えてください.長文で拙いですが,できる限り詳細に記載しました.

\end{document}